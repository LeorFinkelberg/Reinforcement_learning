\documentclass[%
	11pt,
	a4paper,
	utf8,
	%twocolumn
		]{article}	

\usepackage{style_packages/podvoyskiy_article_extended}


\begin{document}
\title{Общие и специальные вопросы оптимизации}

\author{\itshape Подвойский А.О.}

\date{}
\maketitle

\thispagestyle{fancy}

%Здесь приводятся заметки по специальным вопросам теории оптимизации


\shorttableofcontents{Краткое содержание}{1}

\tableofcontents

\section{Полезные ссылки}

\section{Задача обучения с подкреплением}

Под желаемым результатом мы будем понимать максимизацию некоторой скалярной величины, называемой \emph{наградой} (reward). Интеллектуальную сущность (систему/робота/алгоритм), принимающую решения, будем называть \emph{агентом} (agent).

Агент взаимодействует с миром или \emph{средой} (environment), которая задается зависящем от времени \emph{состоянием} (state). Агенту в каждый момент времени в общем случае доступно только некоторое \emph{наблюдение} (observation) текущего состояния мира. Сам агент задает процедуру выбора \emph{действия} (action) по доступным наблюдениям; эту процедуру далее будем называть стратегией или политикой (policy). Процесс взаимодействия агента и среды задается динамикой среды (world dynamics), определяющей правила смены состояний среды во времени и генерации награды.

Буквы $ s $, $ a $, $ r $ зарезервируем для состояний, действий и наград соответственно. Буквой $ t $ будем обозначать время в процессе взаимодействия.

\subsection{Модель взаимодействия агента со средой}





\listoffigures\addcontentsline{toc}{section}{Список иллюстраций}

% Источники в "Газовой промышленности" нумеруются по мере упоминания 
\begin{thebibliography}{99}\addcontentsline{toc}{section}{Список литературы}
	\bibitem{achterberg:constr_int_prog}{\emph{Achterberg T.} Constraint Integer Programming, 2007 }
	
	\bibitem{panteleev}{\emph{Пантлеев А. В., Летова Т.А,} Методы оптимизации в примерах и задачах. -- СПб.: Издательство <<Лань>>, 2015. -- 512 с.}
	
	\bibitem{vorontsova:convex_opt-2021}{\emph{Вороноцова Е.А.} Выпуклая оптимизация. -- М.: МФТИ, 2021. -- 364 с.}
	
	\bibitem{burkov:2020}{\emph{Бурков А.} Машинное обучение без лишних слов. -- СПб.: Питер, 2020. -- 192 с.}
		
	\bibitem{beazley:python-2010}{\emph{Бизли Д.} Python. Подробный справочник. -- Пер. с англ. -- СПб.: Символ-Плюс, 2010. -- 864~с. }
\end{thebibliography}

\end{document}
